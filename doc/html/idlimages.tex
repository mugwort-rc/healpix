\batchmode


\documentclass[12pt,twoside]{article}
\RequirePackage{ifthen}


\usepackage{healpix,graphicx,html,makeidx}
%
\renewcommand{\ell}{l}
%
\renewcommand{\lq}{'}%
\providecommand{\hpxversion}{2.11c}%
\providecommand{\idlversion}{6.0 }%
\providecommand{\nside}{{N_{\rm side}}}%
\providecommand{\npix}{{N_{\rm pix}}}%
\providecommand{\ntemplate}{{N_{\rm template}}}%
\providecommand{\myhtmlimage}[1]{ } 
%
\renewcommand{\contentsname}{{TABLE OF CONTENTS}}


\sloppy
\setcounter{secnumdepth}{0}
\setcounter{tocdepth}{10}%
\providecommand{\facname}{}%
\providecommand{\FACNAME}{} 




\usepackage[dvips]{color}


\pagecolor[gray]{.7}

\usepackage[latin1]{inputenc}



\makeatletter

\makeatletter
\count@=\the\catcode`\_ \catcode`\_=8 
\newenvironment{tex2html_wrap}{}{}%
\catcode`\<=12\catcode`\_=\count@
\newcommand{\providedcommand}[1]{\expandafter\providecommand\csname #1\endcsname}%
\newcommand{\renewedcommand}[1]{\expandafter\providecommand\csname #1\endcsname{}%
  \expandafter\renewcommand\csname #1\endcsname}%
\newcommand{\newedenvironment}[1]{\newenvironment{#1}{}{}\renewenvironment{#1}}%
\let\newedcommand\renewedcommand
\let\renewedenvironment\newedenvironment
\makeatother
\let\mathon=$
\let\mathoff=$
\ifx\AtBeginDocument\undefined \newcommand{\AtBeginDocument}[1]{}\fi
\newbox\sizebox
\setlength{\hoffset}{0pt}\setlength{\voffset}{0pt}
\addtolength{\textheight}{\footskip}\setlength{\footskip}{0pt}
\addtolength{\textheight}{\topmargin}\setlength{\topmargin}{0pt}
\addtolength{\textheight}{\headheight}\setlength{\headheight}{0pt}
\addtolength{\textheight}{\headsep}\setlength{\headsep}{0pt}
\setlength{\textwidth}{349pt}
\newwrite\lthtmlwrite
\makeatletter
\let\realnormalsize=\normalsize
\global\topskip=2sp
\def\preveqno{}\let\real@float=\@float \let\realend@float=\end@float
\def\@float{\let\@savefreelist\@freelist\real@float}
\def\liih@math{\ifmmode$\else\bad@math\fi}
\def\end@float{\realend@float\global\let\@freelist\@savefreelist}
\let\real@dbflt=\@dbflt \let\end@dblfloat=\end@float
\let\@largefloatcheck=\relax
\let\if@boxedmulticols=\iftrue
\def\@dbflt{\let\@savefreelist\@freelist\real@dbflt}
\def\adjustnormalsize{\def\normalsize{\mathsurround=0pt \realnormalsize
 \parindent=0pt\abovedisplayskip=0pt\belowdisplayskip=0pt}%
 \def\phantompar{\csname par\endcsname}\normalsize}%
\def\lthtmltypeout#1{{\let\protect\string \immediate\write\lthtmlwrite{#1}}}%
\newcommand\lthtmlhboxmathA{\adjustnormalsize\setbox\sizebox=\hbox\bgroup\kern.05em }%
\newcommand\lthtmlhboxmathB{\adjustnormalsize\setbox\sizebox=\hbox to\hsize\bgroup\hfill }%
\newcommand\lthtmlvboxmathA{\adjustnormalsize\setbox\sizebox=\vbox\bgroup %
 \let\ifinner=\iffalse \let\)\liih@math }%
\newcommand\lthtmlboxmathZ{\@next\next\@currlist{}{\def\next{\voidb@x}}%
 \expandafter\box\next\egroup}%
\newcommand\lthtmlmathtype[1]{\gdef\lthtmlmathenv{#1}}%
\newcommand\lthtmllogmath{\dimen0\ht\sizebox \advance\dimen0\dp\sizebox
  \ifdim\dimen0>.95\vsize
   \lthtmltypeout{%
*** image for \lthtmlmathenv\space is too tall at \the\dimen0, reducing to .95 vsize ***}%
   \ht\sizebox.95\vsize \dp\sizebox\z@ \fi
  \lthtmltypeout{l2hSize %
:\lthtmlmathenv:\the\ht\sizebox::\the\dp\sizebox::\the\wd\sizebox.\preveqno}}%
\newcommand\lthtmlfigureA[1]{\let\@savefreelist\@freelist
       \lthtmlmathtype{#1}\lthtmlvboxmathA}%
\newcommand\lthtmlpictureA{\bgroup\catcode`\_=8 \lthtmlpictureB}%
\newcommand\lthtmlpictureB[1]{\lthtmlmathtype{#1}\egroup
       \let\@savefreelist\@freelist \lthtmlhboxmathB}%
\newcommand\lthtmlpictureZ[1]{\hfill\lthtmlfigureZ}%
\newcommand\lthtmlfigureZ{\lthtmlboxmathZ\lthtmllogmath\copy\sizebox
       \global\let\@freelist\@savefreelist}%
\newcommand\lthtmldisplayA{\bgroup\catcode`\_=8 \lthtmldisplayAi}%
\newcommand\lthtmldisplayAi[1]{\lthtmlmathtype{#1}\egroup\lthtmlvboxmathA}%
\newcommand\lthtmldisplayB[1]{\edef\preveqno{(\theequation)}%
  \lthtmldisplayA{#1}\let\@eqnnum\relax}%
\newcommand\lthtmldisplayZ{\lthtmlboxmathZ\lthtmllogmath\lthtmlsetmath}%
\newcommand\lthtmlinlinemathA{\bgroup\catcode`\_=8 \lthtmlinlinemathB}
\newcommand\lthtmlinlinemathB[1]{\lthtmlmathtype{#1}\egroup\lthtmlhboxmathA
  \vrule height1.5ex width0pt }%
\newcommand\lthtmlinlineA{\bgroup\catcode`\_=8 \lthtmlinlineB}%
\newcommand\lthtmlinlineB[1]{\lthtmlmathtype{#1}\egroup\lthtmlhboxmathA}%
\newcommand\lthtmlinlineZ{\egroup\expandafter\ifdim\dp\sizebox>0pt %
  \expandafter\centerinlinemath\fi\lthtmllogmath\lthtmlsetinline}
\newcommand\lthtmlinlinemathZ{\egroup\expandafter\ifdim\dp\sizebox>0pt %
  \expandafter\centerinlinemath\fi\lthtmllogmath\lthtmlsetmath}
\newcommand\lthtmlindisplaymathZ{\egroup %
  \centerinlinemath\lthtmllogmath\lthtmlsetmath}
\def\lthtmlsetinline{\hbox{\vrule width.1em \vtop{\vbox{%
  \kern.1em\copy\sizebox}\ifdim\dp\sizebox>0pt\kern.1em\else\kern.3pt\fi
  \ifdim\hsize>\wd\sizebox \hrule depth1pt\fi}}}
\def\lthtmlsetmath{\hbox{\vrule width.1em\kern-.05em\vtop{\vbox{%
  \kern.1em\kern0.8 pt\hbox{\hglue.17em\copy\sizebox\hglue0.8 pt}}\kern.3pt%
  \ifdim\dp\sizebox>0pt\kern.1em\fi \kern0.8 pt%
  \ifdim\hsize>\wd\sizebox \hrule depth1pt\fi}}}
\def\centerinlinemath{%
  \dimen1=\ifdim\ht\sizebox<\dp\sizebox \dp\sizebox\else\ht\sizebox\fi
  \advance\dimen1by.5pt \vrule width0pt height\dimen1 depth\dimen1 
 \dp\sizebox=\dimen1\ht\sizebox=\dimen1\relax}

\def\lthtmlcheckvsize{\ifdim\ht\sizebox<\vsize 
  \ifdim\wd\sizebox<\hsize\expandafter\hfill\fi \expandafter\vfill
  \else\expandafter\vss\fi}%
\providecommand{\selectlanguage}[1]{}%
\makeatletter \tracingstats = 1 


\begin{document}
\pagestyle{empty}\thispagestyle{empty}\lthtmltypeout{}%
\lthtmltypeout{latex2htmlLength hsize=\the\hsize}\lthtmltypeout{}%
\lthtmltypeout{latex2htmlLength vsize=\the\vsize}\lthtmltypeout{}%
\lthtmltypeout{latex2htmlLength hoffset=\the\hoffset}\lthtmltypeout{}%
\lthtmltypeout{latex2htmlLength voffset=\the\voffset}\lthtmltypeout{}%
\lthtmltypeout{latex2htmlLength topmargin=\the\topmargin}\lthtmltypeout{}%
\lthtmltypeout{latex2htmlLength topskip=\the\topskip}\lthtmltypeout{}%
\lthtmltypeout{latex2htmlLength headheight=\the\headheight}\lthtmltypeout{}%
\lthtmltypeout{latex2htmlLength headsep=\the\headsep}\lthtmltypeout{}%
\lthtmltypeout{latex2htmlLength parskip=\the\parskip}\lthtmltypeout{}%
\lthtmltypeout{latex2htmlLength oddsidemargin=\the\oddsidemargin}\lthtmltypeout{}%
\makeatletter
\if@twoside\lthtmltypeout{latex2htmlLength evensidemargin=\the\evensidemargin}%
\else\lthtmltypeout{latex2htmlLength evensidemargin=\the\oddsidemargin}\fi%
\lthtmltypeout{}%
\makeatother
\setcounter{page}{1}
\onecolumn

% !!! IMAGES START HERE !!!

\setcounter{secnumdepth}{0}
\setcounter{tocdepth}{10}
\stepcounter{section}
\stepcounter{section}
{\newpage\clearpage
\lthtmlinlinemathA{tex2html_wrap_inline209}%
$3.46\ 10^{18}$%
\lthtmlinlinemathZ
\lthtmlcheckvsize\clearpage}

\stepcounter{section}
\stepcounter{section}

\renewcommand{\facname}{{ang2vec }}

\renewcommand{\FACNAME}{{ANG2VEC }}
\stepcounter{section}
{\newpage\clearpage
\lthtmlinlinemathA{tex2html_wrap_inline531}%
$\pi$%
\lthtmlinlinemathZ
\lthtmlcheckvsize\clearpage}

{\newpage\clearpage
\lthtmlinlinemathA{tex2html_wrap_inline533}%
$2\pi$%
\lthtmlinlinemathZ
\lthtmlcheckvsize\clearpage}

{\newpage\clearpage
\lthtmlinlinemathA{tex2html_wrap_inline539}%
$x(0),\ldots,x(n-1),\ y(0),\ldots,y(n-1),\ z(0),\ldots,z(n-1)$%
\lthtmlinlinemathZ
\lthtmlcheckvsize\clearpage}

{\newpage\clearpage
\lthtmlinlinemathA{tex2html_wrap_inline541}%
$(\theta,\phi)$%
\lthtmlinlinemathZ
\lthtmlcheckvsize\clearpage}

{\newpage\clearpage
\lthtmlinlinemathA{tex2html_wrap_inline545}%
$x = \sin\theta\cos\phi$%
\lthtmlinlinemathZ
\lthtmlcheckvsize\clearpage}

{\newpage\clearpage
\lthtmlinlinemathA{tex2html_wrap_inline547}%
$y=\sin\theta\sin\phi$%
\lthtmlinlinemathZ
\lthtmlcheckvsize\clearpage}

{\newpage\clearpage
\lthtmlinlinemathA{tex2html_wrap_inline549}%
$z=\cos\theta$%
\lthtmlinlinemathZ
\lthtmlcheckvsize\clearpage}


\renewcommand{\facname}{{bin\_llcl}}

\renewcommand{\FACNAME}{{BIN\_LLCL}}
\stepcounter{section}
{\newpage\clearpage
\lthtmlinlinemathA{tex2html_wrap_inline663}%
$\ldots$%
\lthtmlinlinemathZ
\lthtmlcheckvsize\clearpage}

{\newpage\clearpage
\lthtmlinlinemathA{tex2html_wrap_inline673}%
$C(b) \sqrt{ 2 / ((2l_b+1) \Delta l_b)}$%
\lthtmlinlinemathZ
\lthtmlcheckvsize\clearpage}

{\newpage\clearpage
\lthtmlinlinemathA{tex2html_wrap_inline675}%
$\Delta l(b)$%
\lthtmlinlinemathZ
\lthtmlcheckvsize\clearpage}

{\newpage\clearpage
\lthtmlinlinemathA{tex2html_wrap_inline679}%
$l(l+1)/2\pi$%
\lthtmlinlinemathZ
\lthtmlcheckvsize\clearpage}

{\newpage\clearpage
\lthtmlinlinemathA{tex2html_wrap_inline683}%
$\propto 2l+1$%
\lthtmlinlinemathZ
\lthtmlcheckvsize\clearpage}

{\newpage\clearpage
\lthtmlinlinemathA{tex2html_wrap_inline685}%
$\l (\l +1)/2\pi$%
\lthtmlinlinemathZ
\lthtmlcheckvsize\clearpage}

\stepcounter{section}
{\newpage\clearpage
\lthtmlinlinemathA{tex2html_wrap_inline811}%
$1\le n \le 3$%
\lthtmlinlinemathZ
\lthtmlcheckvsize\clearpage}

\stepcounter{section}
\stepcounter{section}
{\newpage\clearpage
\lthtmlinlinemathA{tex2html_wrap_inline967}%
$\quad$%
\lthtmlinlinemathZ
\lthtmlcheckvsize\clearpage}


\renewcommand{\facname}{{change\_polcconv }}

\renewcommand{\FACNAME}{{CHANGE\_POLCCONV }}
\stepcounter{section}
\stepcounter{section}
\stepcounter{section}

\renewcommand{\facname}{{euler\_matrix\_new}}

\renewcommand{\FACNAME}{{EULER\_MATRIX\_NEW}}
\stepcounter{section}
{\newpage\clearpage
\lthtmlinlinemathA{tex2html_wrap_inline13195}%
$\textstyle \parbox{\hsize}{\facname \ ~\ allows the generation of a rotation Euler matrix. The user
can choose the three Euler angles, and the three axes of rotation.
\par
If vec is an N$\times$3 array containing N 3D vectors, \\
    vecr = vec  \# euler\_matrix\_new(a1,a2,a3,/Y) \\
will be the rotated vectors\\
\par
This routine supersedes euler\_matrix, which had inconsistent angle
definitions. The relation between the two routines is as follows  :
\\[.2cm]
euler\_matrix\_new(a,b,c,/X)  =  euler\_matrix($-$a,$-$b,$-$c,/X) \\
= Transpose(euler\_matrix(c, b, a,/X)) \\[.2cm]
euler\_matrix\_new(a,b,c,/Y)  =  euler\_matrix($-$a, b,$-$c,/Y) \\
= Transpose(euler\_matrix(c,$-$b, a,/Y)) \\[.2cm]
euler\_matrix\_new(a,b,c,/Z)  =  euler\_matrix($-$a, b,$-$c,/Z)
}$%
\lthtmlinlinemathZ
\lthtmlcheckvsize\clearpage}

\stepcounter{section}
\stepcounter{section}
{\newpage\clearpage
\lthtmlinlinemathA{tex2html_wrap_inline1630}%
$C(l) = \sum_m a_{lm}a^*_{lm} / (2l+1)$%
\lthtmlinlinemathZ
\lthtmlcheckvsize\clearpage}

{\newpage\clearpage
\lthtmlinlinemathA{tex2html_wrap_inline1644}%
$l(l+1)C(l)/2\pi$%
\lthtmlinlinemathZ
\lthtmlcheckvsize\clearpage}


\renewcommand{\facname}{{gaussbeam }}

\renewcommand{\FACNAME}{{GAUSSBEAM }}

\renewcommand{\l}{{$l$\  }}
\stepcounter{section}
{\newpage\clearpage
\lthtmlinlinemathA{tex2html_wrap_inline1738}%
$2 \le$%
\lthtmlinlinemathZ
\lthtmlcheckvsize\clearpage}

{\newpage\clearpage
\lthtmlinlinemathA{tex2html_wrap_inline1740}%
$\le 4$%
\lthtmlinlinemathZ
\lthtmlcheckvsize\clearpage}

{\newpage\clearpage
\lthtmlinlinemathA{tex2html_wrap_inline1746}%
$C(l)_{\rm meas} = C(l)
B(l)^2$%
\lthtmlinlinemathZ
\lthtmlcheckvsize\clearpage}


\renewcommand{\facname}{{getdisc\_ring }}

\renewcommand{\FACNAME}{{GETDISC\_RING }}
\stepcounter{section}

\renewcommand{\facname}{{getsize\_fits }}

\renewcommand{\FACNAME}{{GETSIZE\_FITS }}
\stepcounter{section}
{\newpage\clearpage
\lthtmlinlinemathA{tex2html_wrap_inline1904}%
${N_{\rm side}}$%
\lthtmlinlinemathZ
\lthtmlcheckvsize\clearpage}

{\newpage\clearpage
\lthtmlinlinemathA{tex2html_wrap_inline1908}%
$=12{N_{\rm side}}^2$%
\lthtmlinlinemathZ
\lthtmlcheckvsize\clearpage}

{\newpage\clearpage
\lthtmlinlinemathA{tex2html_wrap_inline1912}%
$\le 12{N_{\rm side}}^2$%
\lthtmlinlinemathZ
\lthtmlcheckvsize\clearpage}

{\newpage\clearpage
\lthtmlinlinemathA{tex2html_wrap_inline13285}%
$\textstyle \parbox{\hsize}{ should produce something like \\
   {\em 196608 \ \        128 \ \         256  \ \      2} \\
meaning that the map contained in that file has 196608 pixels, the resolution parameter is
nside=128, the maximum multipole was 256, and this a full sky map
(type 2).
}$%
\lthtmlinlinemathZ
\lthtmlcheckvsize\clearpage}

\stepcounter{section}
\stepcounter{section}
{\newpage\clearpage
\lthtmlinlinemathA{tex2html_wrap_inline2111}%
$\pm$%
\lthtmlinlinemathZ
\lthtmlcheckvsize\clearpage}

{\newpage\clearpage
\lthtmlinlinemathA{tex2html_wrap_inline2113}%
$\mu$%
\lthtmlinlinemathZ
\lthtmlcheckvsize\clearpage}


\renewcommand{\facname}{{healpixwindow }}

\renewcommand{\FACNAME}{{HEALPIXWINDOW }}

\renewcommand{\l}{{$l$\  }}
\stepcounter{section}
{\newpage\clearpage
\lthtmlinlinemathA{tex2html_wrap_inline2216}%
$ \sqrt{3/\pi}\ 3600/N_{\rm side}$%
\lthtmlinlinemathZ
\lthtmlcheckvsize\clearpage}

{\newpage\clearpage
\lthtmlinlinemathA{tex2html_wrap_inline2220}%
$C(l)_{\rm pix} = C(l)
W(l)^2$%
\lthtmlinlinemathZ
\lthtmlcheckvsize\clearpage}

\stepcounter{section}


\newedenvironment{qualifiers_hpx2gs}{\rule{\hsize}{0.7mm}
     \textsc{\Large{\textbf{QUALIFIERS}}}\hfill\newline \renewcommand {\arraystretch}{1.5}%
	}%


%
\providecommand{\sizeoneg}{0.12\hsize}%


%
\providecommand{\sizethrg}{0.85\hsize}%

\stepcounter{section}
{\newpage\clearpage
\lthtmlinlinemathA{tex2html_wrap_inline2502}%
$\backslash$%
\lthtmlinlinemathZ
\lthtmlcheckvsize\clearpage}

\stepcounter{section}
\stepcounter{section}
{\newpage\clearpage
\lthtmlinlinemathA{tex2html_wrap_inline2793}%
$N_{\rm side}$%
\lthtmlinlinemathZ
\lthtmlcheckvsize\clearpage}

\stepcounter{section}
{\newpage\clearpage
\lthtmlinlinemathA{tex2html_wrap_inline2933}%
$0 \le |m|\le l$%
\lthtmlinlinemathZ
\lthtmlcheckvsize\clearpage}


\renewcommand{\facname}{{init\_healpix }}

\renewcommand{\FACNAME}{{INIT\_HEALPIX }}
\stepcounter{section}
\stepcounter{section}

\renewcommand{\facname}{{median\_filter }}

\renewcommand{\FACNAME}{{MEDIAN\_FILTER }}
\stepcounter{section}
\stepcounter{section}
\stepcounter{section}


\newedenvironment{qualifiers_mollview}{\rule{\hsize}{0.7mm}
     \textsc{\Large{\textbf{QUALIFIERS}}}\hfill\newline \renewcommand {\arraystretch}{1.5}%
	}%



\newedenvironment{keywords_mollview}{\rule{\hsize}{0.7mm}
     \textsc{\Large{\textbf{KEYWORDS}}}\hfill\newline \renewcommand {\arraystretch}{1.5}%
	}%


%
\providecommand{\sizeone}{0.19\hsize}%


%
\providecommand{\sizetwo}{0.08\hsize}%


%
\providecommand{\sizethr}{0.70\hsize}%

{\newpage\clearpage
\lthtmlinlinemathA{tex2html_wrap_inline3909}%
$\sinh^{-1}$%
\lthtmlinlinemathZ
\lthtmlcheckvsize\clearpage}

{\newpage\clearpage
\lthtmlinlinemathA{tex2html_wrap_inline3927}%
$\simeq$%
\lthtmlinlinemathZ
\lthtmlcheckvsize\clearpage}

{\newpage\clearpage
\lthtmlinlinemathA{tex2html_wrap_inline3941}%
$\leq 0$%
\lthtmlinlinemathZ
\lthtmlcheckvsize\clearpage}

{\newpage\clearpage
\lthtmlinlinemathA{tex2html_wrap_inline3945}%
$P = \sqrt{\left(U^2 + Q^2\right)}$%
\lthtmlinlinemathZ
\lthtmlcheckvsize\clearpage}

{\newpage\clearpage
\lthtmlinlinemathA{tex2html_wrap_inline3947}%
$\phi = \tan^{-1}(U/Q) /2$%
\lthtmlinlinemathZ
\lthtmlcheckvsize\clearpage}

{\newpage\clearpage
\lthtmlinlinemathA{tex2html_wrap_inline3961}%
$\quad\quad$%
\lthtmlinlinemathZ
\lthtmlcheckvsize\clearpage}


\renewcommand{\facname}{{neighbours\_nest }}
\stepcounter{section}

\renewcommand{\facname}{{neighbours\_ring }}
\stepcounter{section}

\renewcommand{\facname}{{npix2nside }}

\renewcommand{\FACNAME}{{NPIX2NSIDE }}
\stepcounter{section}
{\newpage\clearpage
\lthtmlinlinemathA{tex2html_wrap_inline4524}%
$N_{\rm pix} = 12N_{\rm side}^2$%
\lthtmlinlinemathZ
\lthtmlcheckvsize\clearpage}

{\newpage\clearpage
\lthtmlinlinemathA{tex2html_wrap_inline4528}%
$\{1,\ldots,2^{29}\}$%
\lthtmlinlinemathZ
\lthtmlcheckvsize\clearpage}


\renewcommand{\facname}{{nside2npix }}

\renewcommand{\FACNAME}{{NSIDE2NPIX }}
\stepcounter{section}
{\newpage\clearpage
\lthtmlinlinemathA{tex2html_wrap_inline4666}%
$ \le 2^{29}$%
\lthtmlinlinemathZ
\lthtmlcheckvsize\clearpage}


\renewcommand{\facname}{{nside2ntemplates }}

\renewcommand{\FACNAME}{{NSIDE2NTEMPLATES }}
\stepcounter{section}
{\newpage\clearpage
\lthtmlinlinemathA{tex2html_wrap_inline4778}%
$\{1,\ldots,8192\}$%
\lthtmlinlinemathZ
\lthtmlcheckvsize\clearpage}

{\newpage\clearpage
\lthtmldisplayA{displaymath4780}%
\begin{displaymath}{N_{\rm template}}=\frac{1+{N_{\rm side}}({N_{\rm side}}+6)}{4}.\end{displaymath}%
\lthtmldisplayZ
\lthtmlcheckvsize\clearpage}

\stepcounter{section}
\stepcounter{section}

\renewcommand{\facname}{{pix2xxx, ang2xxx,... }}

\renewcommand{\FACNAME}{{PIX2XXX, ANG2XXX,...}}
\stepcounter{section}
{\newpage\clearpage
\lthtmlinlinemathA{tex2html_wrap_inline5156}%
$x_N(0),\ldots,x_N(n-1),\ y_N(0),\ldots,y_N(n-1),\ z_N(0),\ldots,z_N(n-1)$%
\lthtmlinlinemathZ
\lthtmlcheckvsize\clearpage}

{\newpage\clearpage
\lthtmlinlinemathA{tex2html_wrap_inline5158}%
$x_W(0),\ldots,x_W(n-1),\ y_W(0),\ldots,y_W(n-1),\
                   z_W(0),\ldots,z_W(n-1)$%
\lthtmlinlinemathZ
\lthtmlcheckvsize\clearpage}


\renewcommand{\facname}{{query\_disc }}

\renewcommand{\FACNAME}{{QUERY\_DISC }}
\stepcounter{section}

\renewcommand{\facname}{{query\_polygon }}

\renewcommand{\FACNAME}{{QUERY\_POLYGON }}
\stepcounter{section}

\renewcommand{\facname}{{query\_strip }}

\renewcommand{\FACNAME}{{QUERY\_STRIP }}
\stepcounter{section}
{\newpage\clearpage
\lthtmlinlinemathA{tex2html_wrap_inline5537}%
$\pi/5$%
\lthtmlinlinemathZ
\lthtmlcheckvsize\clearpage}

{\newpage\clearpage
\lthtmlinlinemathA{tex2html_wrap_inline5539}%
$3\pi/4$%
\lthtmlinlinemathZ
\lthtmlcheckvsize\clearpage}


\renewcommand{\facname}{{query\_triangle }}

\renewcommand{\FACNAME}{{QUERY\_TRIANGLE }}
\stepcounter{section}

\renewcommand{\facname}{{read\_fits\_cut4 }}

\renewcommand{\FACNAME}{{READ\_FITS\_CUT4 }}
\stepcounter{section}
{\newpage\clearpage
\lthtmlinlinemathA{tex2html_wrap_inline5733}%
$\propto 1/\sqrt{{\rm n\_obs}}$%
\lthtmlinlinemathZ
\lthtmlcheckvsize\clearpage}


\renewcommand{\facname}{{read\_fits\_map }}

\renewcommand{\FACNAME}{{READ\_FITS\_MAP }}
\stepcounter{section}
{\newpage\clearpage
\lthtmlinlinemathA{tex2html_wrap_inline5860}%
$\ge$%
\lthtmlinlinemathZ
\lthtmlcheckvsize\clearpage}


\renewcommand{\facname}{{read\_fits\_s }}

\renewcommand{\FACNAME}{{READ\_FITS\_S }}
\stepcounter{section}

\renewcommand{\facname}{{read\_tqu }}

\renewcommand{\FACNAME}{{READ\_TQU }}
\stepcounter{section}
{\newpage\clearpage
\lthtmlinlinemathA{tex2html_wrap_inline6099}%
$N_{\rm pix}$%
\lthtmlinlinemathZ
\lthtmlcheckvsize\clearpage}

{\newpage\clearpage
\lthtmlinlinemathA{tex2html_wrap_inline6103}%
$\le$%
\lthtmlinlinemathZ
\lthtmlcheckvsize\clearpage}


\renewcommand{\facname}{{remove\_dipole}}

\renewcommand{\FACNAME}{{REMOVE\_DIPOLE}}
\stepcounter{section}
{\newpage\clearpage
\lthtmlinlinemathA{tex2html_wrap_inline6226}%
$\equiv -1.6375\ 10^{30}$%
\lthtmlinlinemathZ
\lthtmlcheckvsize\clearpage}


\renewcommand{\facname}{{reorder }}

\renewcommand{\FACNAME}{{REORDER }}
\stepcounter{section}

\renewcommand{\facname}{{rotate\_coord}}

\renewcommand{\FACNAME}{{ROTATE\_COORD}}
\stepcounter{section}
\stepcounter{section}
{\newpage\clearpage
\lthtmlinlinemathA{tex2html_wrap_indisplay14010}%
$\displaystyle z=\cos(\theta)\ge 2/3,$%
\lthtmlindisplaymathZ
\lthtmlcheckvsize\clearpage}

{\newpage\clearpage
\lthtmlinlinemathA{tex2html_wrap_indisplay14011}%
$\textstyle \ $%
\lthtmlindisplaymathZ
\lthtmlcheckvsize\clearpage}

{\newpage\clearpage
\lthtmlinlinemathA{tex2html_wrap_indisplay14012}%
$\displaystyle 0< \phi \leq \pi/2,$%
\lthtmlindisplaymathZ
\lthtmlcheckvsize\clearpage}

{\newpage\clearpage
\lthtmlinlinemathA{tex2html_wrap_indisplay14013}%
$\displaystyle %[Nside*(Nside+2)/4]
     2/3 > z \geq 0,$%
\lthtmlindisplaymathZ
\lthtmlcheckvsize\clearpage}

{\newpage\clearpage
\lthtmlinlinemathA{tex2html_wrap_indisplay14015}%
$\displaystyle \phi=0, \quad{\rm or}\quad  \phi=\frac{\pi}{4{N_{\rm side}}}.   %[Nside]
$%
\lthtmlindisplaymathZ
\lthtmlcheckvsize\clearpage}

{\newpage\clearpage
\lthtmlinlinemathA{tex2html_wrap_inline6553}%
$\phi$%
\lthtmlinlinemathZ
\lthtmlcheckvsize\clearpage}

{\newpage\clearpage
\lthtmlinlinemathA{tex2html_wrap_inline6557}%
${N_{\rm side}}=256$%
\lthtmlinlinemathZ
\lthtmlcheckvsize\clearpage}

\stepcounter{section}
{\newpage\clearpage
\lthtmlinlinemathA{tex2html_wrap_inline6693}%
$12{N_{\rm side}}^2-1$%
\lthtmlinlinemathZ
\lthtmlcheckvsize\clearpage}

{\newpage\clearpage
\lthtmlinlinemathA{tex2html_wrap_indisplay14031}%
$\displaystyle %[Nside*(Nside+2)/4]
     2/3 > z \geq 0,$%
\lthtmlindisplaymathZ
\lthtmlcheckvsize\clearpage}

{\newpage\clearpage
\lthtmlinlinemathA{tex2html_wrap_indisplay14033}%
$\displaystyle \phi=0, \quad{\rm or}\quad  \phi=\frac{\pi}{4{N_{\rm side}}}.   %[Nside]
$%
\lthtmlindisplaymathZ
\lthtmlcheckvsize\clearpage}


\renewcommand{\facname}{{ud\_grade }}

\renewcommand{\FACNAME}{{UD\_GRADE }}
\stepcounter{section}
{\newpage\clearpage
\lthtmlinlinemathA{tex2html_wrap_inline14043}%
$\textstyle \parbox{0.5\hsize}{if set, during {\bf degradation} each big pixel containing one
    bad or missing small pixel is also considered as bad, \\
        if not set, each big pixel containing at least one good pixel
    is considered as good (optimistic)
       default = 0 (:not set)}$%
\lthtmlinlinemathZ
\lthtmlcheckvsize\clearpage}


\renewcommand{\facname}{{vec2ang }}

\renewcommand{\FACNAME}{{VEC2ANG }}
\stepcounter{section}

\renewcommand{\facname}{{write\_fits\_cut4 }}

\renewcommand{\FACNAME}{{WRITE\_FITS\_CUT4 }}
\stepcounter{section}
{\newpage\clearpage
\lthtmlinlinemathA{tex2html_wrap_inline7164}%
${N_{\rm side}}=32$%
\lthtmlinlinemathZ
\lthtmlcheckvsize\clearpage}


\renewcommand{\facname}{{write\_fits\_map }}

\renewcommand{\FACNAME}{{WRITE\_FITS\_MAP }}
\stepcounter{section}

\renewcommand{\facname}{{write\_fits\_sb }}

\renewcommand{\FACNAME}{{WRITE\_FITS\_SB }}
\stepcounter{section}
{\newpage\clearpage
\lthtmlinlinemathA{tex2html_wrap_inline14110}%
$\textstyle \parbox{\hsize}{\facname writes out the information contained in {\tt{Prim\_stc}} and {\tt{Exten\_stc}} in the primary unit and extension of the FITS file
{\tt File} respectively . Coordinate systems can also be specified by {\tt Coordsys}. Specifying the
ordering scheme is compulsary for \healpix data sets and can be done either in {\tt Header} or by setting {\tt
Ordering} or {\tt Nested} or {\tt Ring} to the correct value. If {\tt
Ordering} or {\tt Nested} or {\tt Ring} is set, its value overrides what is
given in {\tt Header}. \\
\par
The data is assumed to represent a full sky data set with 
the number of data points npix = 12*Nside*Nside
unless   \hfill\newline
Partial is set OR the input fits header contains OBJECT =
               'PARTIAL' \hfill\newline
       AND \hfill\newline
         the Nside qualifier is given a valid value OR the FITS header contains
                 a NSIDE}$%
\lthtmlinlinemathZ
\lthtmlcheckvsize\clearpage}


\renewcommand{\facname}{{write\_tqu }}

\renewcommand{\FACNAME}{{WRITE\_TQU }}
\stepcounter{section}

\end{document}
